%%% Fiktivní kapitola s ukázkami tabulek, obrázků a kódu

\chapter{Řešítko}
Abychom mohli dokončit důkaz věty /TODO odkaz/ nebo alespoň některých jejích instancí, potřebujeme získat požadované stavební bloky. Nabízí se naprogramovat řešítko, které bude umět alespoň některé typy hledaných grafů najít. Hlavním cílem této práce bylo takový program připravit a pomocí něj získat lepší představu o potenciálu uvedené konstrukce důkazu.

Program na vstupu očekává zadání vnější stěny: každý vrchol je zastoupen jedním bitem, který určuje, zda má být ve výsledném grafu stupně 2 nebo 3. K tomu očekává seznam velikostí stěn, které má využít. Na výstupu informuje, zda se mu daný graf podařilo najít, a je schopný jej exportovat.

Postup hledání původně imitoval lidské pokusy o řešení problému: nakreslí si vnější stěnu, zkusí spojit nějaké dva vrcholy řetízkem vhodné délky (aby nově uzavřená stěna byla z neutrální posloupnosti) a dokud má místo na papíře, spojuje. Pak si překreslí nejvnitřnější, zatím neuzavřenou stěnu (budeme mluvit o \textbf{hranici}), ta se stane "vnější stěnou" na novém papíře a pokračuje. Pokud dojde do situace, kdy neumí dál nic spojit, nebo je jasné, že graf nemůže vyhovovat parametrům, vrátí se podle uvážení zpět. Během hledání tedy vůbec není třeba si pamatovat celý rozpracovaný graf, stačí pracovat s hranicemi, které navíc stačí reprezentovat jako binární číslo. Výsledkem by pak mohla být jen posloupnost hranic, kterými se prošlo před uzavřením grafu. Překvapivě obtížné je pak z této posloupnosti nestrojově získat skutečný graf, proto program nabízí i možnost graf dodatečně rekonstruovat podle prošlých stavů.

V tento okamžik je jasné, že problém je vlastně prohledávání v řetězcích (které reprezentují hranice), je proto vhodné zmínit, podle jakého kritéria se program rozhoduje, kterým směrem hledat dále. Implementace vždy upřednostňuje ke zpracování již nalezený řetězec nejmenší délky, a pro něj najde všechny další sousedy.

Pro jistotu poznamenejme, že pokud program hledaný graf nenašel, může, ale nemusí to znamenat, že neexistuje.

Pro dokončení důkazu potřebujeme tyto čtyři typy grafů.
