\chapter{Výsledky} \label{vysledky}

Díky programu popsanému v předchozí kapitole bylo možné zkusit dokončit důkaz Věty \ref{veta02:hypoteza} pro některé neutrální posloupnosti $q$ (na volbě posloupnosti $p$ nezáleží, protože konstrukce pro jádrový triark nezávisí na velikosti jádra). 

Vzhledem k výpočetním omezením programu jsme zkusili dokončit důkaz věty pouze pro takové neutrální posloupnosti $q$, že $Q = \lbrace r, s\rbrace$ a navíc $r<s<18$. Seznam posloupností, pro které program nalezl potřebné grafy je v Tabulce \ref{obr03:tabvysledky}. Pokud bychom se omezili na rozsah $r<s<14$, pak program grafy nalezl právě tehdy, když $r$ a $s$ jsou nesoudělná čísla.

\begin{table}[h]\centering
\begin{tabular}{ c | c c c c c c c c c c c }
  {$r\setminus s$} & 7 & 8 & 9 & 10 & 11 & 12 & 13 & 14 & 15 & 16 & 17 \\ \hline
  3 & $\bullet$ & $\bullet$ &  & $\bullet$ & $\bullet$ &  & $\bullet$ & $\bullet$ &  & $\bullet$ & $\bullet$ \\
  4 & $\bullet$ &  & $\bullet$ &  & $\bullet$ &  & $\bullet$ &  & $\bullet$ \\
  5 & $\bullet$ & $\bullet$ & $\bullet$ &  & $\bullet$ & $\bullet$ & $\bullet$  
\end{tabular}
\caption{Úplný výčet dvojic stran, pro které se podařilo dokončit důkaz Věty \ref{veta02:hypoteza}.}
\label{obr03:tabvysledky}
\end{table}

Formulujme výsledek do věty.

\begin{veta} \label{veta:vysledek}
Pro každou neutrální posloupnost $q$, která má nenulové hodnoty pouze pro $q_r$ a $q_s$, kde $r,s$ je vyznačená dvojice z Tabulky \ref{obr03:tabvysledky}, platí následující: 
mějme přípustnou posloupnost $p=(p_k | 3 \leq k \neq 6)$, pak existuje nekonečně takových přirozených $n$, že $p+nq$ je realizovatelná.
\end{veta}


\begin{table}[h]\centering
\begin{tabular}{| c || c | c | c |}
\hline 
{$s\setminus r$} & 3 &4&5 \\ \hline \hline

7 & \cellcolor{lightgray}(i): 3 & \cellcolor{lightgray}(i): 3 & \cellcolor{lightgray}(i): 3\\
 & \cellcolor{lightgray}(ii): 3 & \cellcolor{lightgray}(ii): 3 & \cellcolor{lightgray}(ii): 3\\
 & \cellcolor{lightgray}AB : 101010 & \cellcolor{lightgray}AB : 101010 & \cellcolor{lightgray}AB : 101010\\
 & \cellcolor{lightgray}CD : 101010 & \cellcolor{lightgray}CD : 101010 & \cellcolor{lightgray}CD : 1010\\\hline

8 & \cellcolor{lightgray}(i): 3 & (i): 3,5,6,9 & \cellcolor{lightgray}(i): 3\\
 & \cellcolor{lightgray}(ii): 3 & (ii):  & \cellcolor{lightgray}(ii): 3\\
 & \cellcolor{lightgray}AB : 1010 & AB :  & \cellcolor{lightgray}AB : 1010\\
 & \cellcolor{lightgray}CD : 1010 & CD :  & \cellcolor{lightgray}CD : 1010\\\hline

9 & (i): 3,4,8 & \cellcolor{lightgray}(i): 3 & \cellcolor{lightgray}(i): 3\\
 & (ii):  & \cellcolor{lightgray}(ii): 3 & \cellcolor{lightgray}(ii): 3\\
 & AB :  & \cellcolor{lightgray}AB : 1010 & \cellcolor{lightgray}AB : 1010\\
 & CD :  & \cellcolor{lightgray}CD : 1010 & \cellcolor{lightgray}CD : 1010\\\hline

10 & \cellcolor{lightgray}(i): 3 & (i): 3,4,5,6,7,8,9,10 & (i): 4,5,9,10\\
 & \cellcolor{lightgray}(ii): 3 & (ii):  & (ii): \\
 & \cellcolor{lightgray}AB : 1010 & AB :  & AB : \\
 & \cellcolor{lightgray}CD : 1010 & CD : 1010 & CD : \\\hline

11 & \cellcolor{lightgray}(i): 3 & \cellcolor{lightgray}(i): 3 & \cellcolor{lightgray}(i): 3\\
 & \cellcolor{lightgray}(ii): 3 & \cellcolor{lightgray}(ii): 3 & \cellcolor{lightgray}(ii): 3\\
 & \cellcolor{lightgray}AB : 1010 & \cellcolor{lightgray}AB : 1010 & \cellcolor{lightgray}AB : 1010\\
 & \cellcolor{lightgray}CD : 1010 & \cellcolor{lightgray}CD : 1010 & \cellcolor{lightgray}CD : 1010\\\hline

12 & (i): 3,4,5,6,7,8,9,10 & (i): 5,6 & \cellcolor{lightgray}(i): 3\\
 & (ii):  & (ii):  & \cellcolor{lightgray}(ii): 3\\
 & AB :  & AB :  & \cellcolor{lightgray}AB : 1010\\
 & CD :  & CD :  & \cellcolor{lightgray}CD : 1010\\\hline

13 & \cellcolor{lightgray}(i): 3 & \cellcolor{lightgray}(i): 3 & \cellcolor{lightgray}(i): 3\\
 & \cellcolor{lightgray}(ii): 3 & \cellcolor{lightgray}(ii): 3 & \cellcolor{lightgray}(ii): 3\\
 & \cellcolor{lightgray}AB : 1010 & \cellcolor{lightgray}AB : 1010 & \cellcolor{lightgray}AB : 1010\\
 & \cellcolor{lightgray}CD : 1010 & \cellcolor{lightgray}CD : 1010 & \cellcolor{lightgray}CD : 1010\\\hline

14 & \cellcolor{lightgray}(i): 3 & (i): 3,4,5,6,7,8,9,10 & (i): \\
 & \cellcolor{lightgray}(ii): 3 & (ii):  & (ii): \\
 & \cellcolor{lightgray}AB : 1010 & AB :  & AB : \\
 & \cellcolor{lightgray}CD : 1010 & CD : 1010 & CD : \\\hline

15 & (i): 3,4,7,8 & \cellcolor{lightgray}(i): 3 & (i): 9,10\\
 & (ii):  & \cellcolor{lightgray}(ii): 3 & (ii): \\
 & AB :  & \cellcolor{lightgray}AB : 1010 & AB : \\
 & CD :  & \cellcolor{lightgray}CD : 1010 & CD : \\\hline

16 & \cellcolor{lightgray}(i): 3 & (i): 3,5,6,8,9 & (i): \\
 & \cellcolor{lightgray}(ii): 3 & (ii):  & (ii): \\
 & \cellcolor{lightgray}AB : 1010 & AB :  & AB : \\
 & \cellcolor{lightgray}CD : 1010 & CD :  & CD : \\\hline

17 & \cellcolor{lightgray}(i): 3 & (i):  & (i): \\
 & \cellcolor{lightgray}(ii): 3 & (ii):  & (ii): \\
 & \cellcolor{lightgray}AB : 1010 & AB :  & AB : \\
 & \cellcolor{lightgray}CD : 1010 & CD :  & CD : \\\hline

\end{tabular}
\caption{Přehled nalezených grafů. Ohraničené buňky kódují výsledek pro danou dvojici $r$, $s$. Pokud jsou grafy nalezené, je buňka šedá. První řádek buňky: $k$ pro nalezené ($k$, $k$, $k$)-triarky; druhý: $k$ pro nalezené ($k$, $k$, $k$)- a zároveň ($k$, $k$, $k-1$)-triarky; třetí: řetízek pro konstrukci grafu (iii), kde $mk$ odpovídá poslední hodnotě v druhém řádku; čtvrtý: řetízek pro graf (iv).}
\label{obr03:tabvysledkycele}
\end{table}

Všechny potřebné grafy pro doložení důkazu jsou k práci přiloženy. TODO přiložit