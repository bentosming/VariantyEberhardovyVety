\chapter{Výsledky} \label{vysledky}

Díky programu popsanému v předchozí kapitole bylo možné zkusit dokončit důkaz Věty \ref{veta02:hypoteza} pro některé neutrální posloupnosti $q$ (na volbě posloupnosti $p$ nezáleží, protože konstrukce pro jádrový triark nezávisí na velikosti jádra). 

Vzhledem k výpočetním omezením programu jsme zkusili dokončit důkaz věty pouze pro takové neutrální posloupnosti $q$, že $Q = \lbrace r, s\rbrace$ a navíc $r<s<18$. Seznam posloupností, pro které program nalezl potřebné grafy je v tabulce \ref{obr03:tabvysledky}. Pokud bychom se omezili na rozsah $r<s<14$, pak program grafy nalezl právě tehdy, když $r$ a $s$ jsou nesoudělná čísla.

\begin{figure}[h]\centering
\begin{tabular}{ c | c c c c c c c c c c c }
  {$r\setminus s$} & 7 & 8 & 9 & 10 & 11 & 12 & 13 & 14 & 15 & 16 & 17 \\ \hline
  3 & $\bullet$ & $\bullet$ &  & $\bullet$ & $\bullet$ &  & $\bullet$ & $\bullet$ &  & $\bullet$ & $\bullet$ \\
  4 & $\bullet$ &  & $\bullet$ &  & $\bullet$ &  & $\bullet$ &  & $\bullet$ \\
  5 & $\bullet$ & $\bullet$ & $\bullet$ &  & $\bullet$ & $\bullet$ & $\bullet$  
\end{tabular}
\caption{Úplný výčet dvojic stran, pro které se podařilo dokončit důkaz Věty \ref{veta02:hypoteza}.}
\label{obr03:tabvysledky}
\end{figure}


TODO poznamenat, že to jsou hodnoty mk

\begin{figure}[h]\centering
\begin{tabular}{| c | c | c | c |}
\hline
{$s\setminus r$} & 3 &4&5 \\ \hline
 7 & \cellcolor{lightgray}Nalezeno & \cellcolor{lightgray}Nalezeno & \cellcolor{lightgray}Nalezeno\\
 & \cellcolor{lightgray}(i): 3 & \cellcolor{lightgray}(i): 3 & \cellcolor{lightgray}(i): 3,4,5,6,7\\
 & \cellcolor{lightgray}(ii): 3 & \cellcolor{lightgray}(ii): 3 & \cellcolor{lightgray}(ii): 3,4,5,6,7\\
 & \cellcolor{lightgray}AB : 101010 & \cellcolor{lightgray}AB : 101010 & \cellcolor{lightgray}AB : 01001101\\
 & \cellcolor{lightgray}CD : 101010 & \cellcolor{lightgray}CD : 101010 & \cellcolor{lightgray}CD : 1010\\\hline

8 & \cellcolor{lightgray}Nalezeno & Nenalezeno & \cellcolor{lightgray}Nalezeno\\
 & \cellcolor{lightgray}(i): 3 & (i): 3,5,6,9 & \cellcolor{lightgray}(i): 3\\
 & \cellcolor{lightgray}(ii): 3 & (ii):  & \cellcolor{lightgray}(ii): 3\\
 & \cellcolor{lightgray}AB : 1010 & AB :  & \cellcolor{lightgray}AB : 1010\\
 & \cellcolor{lightgray}CD : 1010 & CD :  & \cellcolor{lightgray}CD : 1010\\\hline

9 & Nenalezeno & \cellcolor{lightgray}Nalezeno & Nenalezeno\\
 & (i): 3,4,8 & \cellcolor{lightgray}(i): 3 & (i): 3,4,5,6,7,8,9,10\\
 & (ii):  & \cellcolor{lightgray}(ii): 3 & (ii): 3,4,5,6,7,8,9,10\\
 & AB :  & \cellcolor{lightgray}AB : 1010 & AB : \\
 & CD :  & \cellcolor{lightgray}CD : 1010 & CD : \\\hline



\end{tabular}
\caption{Úplný výčet dvojic stran, pro které se podařilo dokončit důkaz Věty \ref{veta02:hypoteza}.}
\label{obr03:tabvysledky}
\end{figure}

TODO fakt větu jako výsledek!!

Všechny potřebné grafy pro doložení důkazu jsou k práci přiloženy. TODO přiložit