%%% Fiktivní kapitola s ukázkami citací

\chapter{Konstrukce důkazu}

Představme hypotézu, kterou se snažíme ověřit.
\begin{veta}\label{vera02:1}
Pro každou přípustnou posloupnost $p=(p_k | 3 \leq k \neq 6)$ a neutrální posloupnost $q=(q_k | 3 \leq k \neq 6)$, existuje takové přirozené $n$, že $p+nq$ je přijatelná.
\end{veta}

TODO zmínit výjimku pro torus.

V článku TODO odkaz autoři naznačují konstrukci důkazu, za předpokladu, že existují nějaké pomocné grafy. V další kapitole představíme způsob, jak takové grafy hledat. Teď se zaměříme na důkaz samotný.

Definujme nejprve graf, který v důkazu pomáhal již Eberhardovi.

\begin{definice}[Triarc]\label{def02:1}
\textbf{Triarc} je takový rovinný graf $T$, že vrcholy jeho vnější stěny tvoří cyklus $C$, každý vnitřní vrchol (tj. vrcholy $T-C$) má v $T$ stupeň právě 3 a v $C$ jsou navzájem různé vrcholy $x$, $y$, $z$ stupně 2 - \textbf{rohy}, že vrcholy každé ze tří cest v C, které vzniknou odstraněním rohů z C, mají střídavě stupeň 2 a 3, počínaje i konče stupněm 2.

\textbf{Strana} triarcu je každá z výše zmíněných cest v C, ke které na oba konce připojíme i příslušný roh.

\textbf{Délka strany} triarcu odpovídá počtu jejích vnitřních vrcholů stupně 2 v T.

O triarcu se stranami délky $a$, $b$, $c$ mluvíme jako o \textbf{($a$, $b$, $c$,)-triarcu}. Poznamenejme, že na pořadí stran v názvu nezáleží (odpovídají rotacím).

Později využijeme ještě dalšího značení. \textbf{$M$-triarc} má vnitřní strany pouze velikostí z $M$.
\end{definice}

Zmiňme velmi užitečnou vlastnost triarců: pokud máme dva triarcy, oba mající stranu stejné délky, můžeme je za tuto stranu slepit a zíkáme opět graf, jehož vnitřní vrcholy mají stupeň 3. (Při slepovaní se ztotožní vždy vrchol stupně 2 s vrcholem stupně 3.) Podle jeho tvaru budeme mluvit o \textbf{rovnoběžníku}.

Mohli bychom ale chtít spojovat (alespoň nějaké) triarcy tak, aby výsledkem byl opět triarc. Mějme ($a_1$, $b_1$, $c_1$)-triarc a ($a_2$, $b_2$, $c_2$)-triarc, kde $b_1$ a $c_2$ jsou sudé, a vhodný rovnoběžník(třeba vzniklý spojením dvou vhodných triarců). Slepením vznikne ($a_1+a_2$, $b_1+b_2$, $c_1+c_2$)triarc. 

TODO prodloužit popis podle výmluvnosti obrázku.

TODO připravit obrázky.

Myšlenka důkazu pak není příliš složitá: každou stěnu ze zadané posloupnosti $p$ zabalíme do triarcu, připravíme si pomocné lepící a zkrášlující prvky, díky kterým získáme jediný velký rovnostranný triarc. K němu zkonstruujeme ještě jeden se stejně dlouhými stranami a pomocným prstencem je spojíme v kýžený graf. Všechny tyto pomocné objekty jsou totiž (kromě středu zabalovacích triarců) jen ze stěn, které jsou v zadané neutrální posloupnosti.

