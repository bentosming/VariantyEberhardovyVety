\chapter*{Úvod}
\addcontentsline{toc}{chapter}{Úvod}

Zkusíme-li upravovat Eulerův vzorec pro rovinné grafy
$|V|-|E|+|S|=2$, můžeme pro kubické grafy dospět do tvaru 
\begin{equation}\label{eq01:neccCond}
\sum_{k \geq 3}{(6-k)p_k}=12,
\end{equation}
kde $p_k$ značí počet $k$-hranných stěn grafu. Výraz \eqref{eq01:neccCond} je nutnou podmínkou pro existenci rovinného kubického grafu s daným počtem stěn. Pozoruhodně, počet šestihranných stěn v této podmínce nehraje žádnou roli. Toho si všiml Eberhard \cite{Eberhard91}, který v roce 1891 formuloval a dokázal, že pokud můžeme volit $p_6$, dokážeme najít kubický graf zadaný počtem ostatních stěn, který je rovinný.

\begin{veta}[Eberhardova věta]\label{veta:Eberhard}
Pro každou posloupnost $(p_k \mid 3 \leq k \neq 6)$ kladných celých čísel, splňující \eqref{eq01:neccCond} existuje taková hodnota $p_6$, že existuje rovinný 3-regulární 3-souvislý graf, který má právě $p_k$ $k$-hranných stěn pro každé $k \in \mathbb{N}, k \geq 3 $.
\end{veta}

Na větu navázali další a dnes je známá celá řada jejích obměn. Fisher \citep{Fisher74} dokázal silnější verzi Eberhardovy věty, kde počet potřebných stěn velikosti 6 shora omezil počtem ostatních stěn chtěného grafu. Grünbaum \citep{Grunbaum} představuje vlastní stručnější důkaz Eberhardovy věty a shrnuje výsledky podobného typu. 

DeVos a kol. \citep{Samal09} představuje obdobu Eberhardovy věty, kde místo $p_6$ můžeme volit $p_5$ a $p_7$. Článek také ukazuje, že konstrukce důkazu, kterou autoři použili, by mohla jít uplatnit i pro další velikosti stěn, které mohou splnit \eqref{eq01:neccCond}. 

V tomto textu nejprve zavedeme potřebnou terminologii a představíme zmiňovanou strategii důkazu, která pro dokončení potřebuje najít pomocné grafy. Cílem práce bylo tyto grafy získat. V závěrečných kapitolách navrhneme program, který bude potřebné grafy hledat, a nakonec představíme, pro které velikosti stěn jsme grafy získali a tedy dokončili důkaz dalších variant Eberhardovy věty.
