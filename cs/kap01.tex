%%% Fiktivní kapitola s ukázkami sazby

\chapter{Pojmy a definice}

Dokažme nejprve, jak z Eulerova vzorce získáme \eqref{eq01:neccCond}. Pro každý kubický graf $G = (V,E)$ platí $3 |V| = 2 |E|$. Obecně platí, že počet stěn $s$ můžeme přepsat následovně $s = \sum_{k \geq 3}{p_k}$ a pro rovinné grafy platí i $ \sum_{k \geq 3}{k \cdot p_k}= 2|E|$, protože pokud pro každou stěnu započteme každou její hranu, získáme právě $2|E|$. Úpravou a dosazením do vzorce získáme požadovaný výraz: 
\begin{align*}
|V|-|E|+|S|=2 \\ \frac{1}{3} |E| + s = 2 \\
\frac{1}{6} \sum_{k \geq 3}{k \cdot p_k} - \sum_{k \geq 3}{p_k} = 2 \\ 
\sum_{k \geq 3}{(6-k)p_k}=12
\end{align*}
Čímž je důkaz dokončen.

Zaveďme nyní základní pojmy. Stěně rovinného grafu, která se skládá z $k$ hran, budeme říkat jednoduše \textbf{$k$-úhelník}. Pokud máme posloupnost $(p_k) = (p_3,p_4,...)$, která bude představovat počty k-úhelníků v grafu, definujme $P = \lbrace k \mid p_k \neq 0 \rbrace$ množinu jejích stěn.

Klasifikujme posloupnosti $(p_k) = (p_3,p_4,...)$ podle jejich vlastností.

\begin{definice}[Neutrální posloupnost]\label{def01:neutralni}
Posloupnost $(p_k) = (p_3,p_4,...)$ nezáporných celých čísel je neutrální, pokud $\sum_{k \geq 3}{(6-k)p_k}=0$.
\end{definice}

\begin{definice}[Přípustná posloupnost]\label{def01:pripustna}
Posloupnost $(p_k) = (p_3,p_4,...)$ nezáporných celých čísel je přípustná, pokud $\sum_{k \geq 3}{(6-k)p_k}=6$.
\end{definice}

\begin{definice}[Přijatelná posloupnost]\label{def01:prijatelna}
Posloupnost $(p_k) = (p_3,p_4,...)$ nezáporných celých čísel je přijatelná, pokud existuje rovinný kubický graf, který má právě $p_k$ $k$-úhelníků.
\end{definice}

Všimněme si, že původní Eberhardova věta, a i mnoho jejích variant, byla formulována pro 3-spojité grafy, tedy přesněji pro jednoduché konvexní 3-polytopy. Bijekci mezi těmito dvěma strukturami dokázal až o několik desítek let později Steinitz.

Pro ujištění pojmů uveďme následující jednoduché vlastnosti přijatelných a neutrálních posloupností, které využijeme později.

\begin{tvrz}\label{veta:posloupnosti}
Pro každou neutrální posloupnost $q$ platí $\exists a,b \in Q : a <6 \wedge b>6$.

Součet dvou neutrálních posloupností je neutrální posloupnost. Součet neutrální a přípustné posloupnosti je přípustná posloupnost.

Pro každou posloupnost $(p_k) = (p_3,p_4,...)$ existuje neutrální posloupnost $q$, že $p+qn$ je přijatelná pro nějaké $n \in \mathbb{N}$.
\end{tvrz}

Pro ukázání první vlastnosti si stačí uvědomit, že chceme $\sum_{k \geq 3}{(6-k)p_k} = 0$, kde pro $k < 6$ je sčítaný člen kladný, zato pro $k>6$ je sčítaný člen záporný. Druhá vlastnost plyne z definice a distributivity násobení. Třetí tvrzení pro $q = (q_6)$ je slabší verze \eqref{veta:Eberhard}.



