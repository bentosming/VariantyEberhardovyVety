%%% Fiktivní kapitola s ukázkami sazby

\chapter{Pojmy a definice}

Pokud čtenář není seznámen se základy teorie grafů, doporučujeme začít třeba knihou Kapitoly z diskrétní matematiky \cite{Matousek}. Před zadefinováním základní terminologie pro tuto práci dokažme nejprve, jak z Eulerova vzorce získáme \eqref{eq01:neccCond}. 

\begin{dukaz}
Pro každý kubický graf $G = (V,E)$ platí $3 |V| = 2 |E|$. Obecně platí, že počet stěn $s$ můžeme přepsat následovně $s = \sum_{k \geq 3}{p_k}$ a pro rovinné grafy platí i $ \sum_{k \geq 3}{k \cdot p_k}= 2|E|$, protože pokud pro každou stěnu započteme každou její hranu, získáme právě $2|E|$. Úpravou a dosazením do vzorce získáme požadovaný výraz: 
\begin{align*}
|V|-|E|+|S|=2 \\ -\frac{1}{3} |E| + s = 2 \\
-\frac{1}{6} \sum_{k \geq 3}{k \cdot p_k} + \sum_{k \geq 3}{p_k} = 2 \\ 
\sum_{k \geq 3}{(6-k)p_k}=12
\end{align*}
\end{dukaz}

Zaveďme nyní základní pojmy. Stěně rovinného grafu, která se skládá z $k$ hran, budeme říkat jednoduše \textbf{$\boldsymbol{k}$-úhelník}. Pokud máme posloupnost $(p_k) = (p_3,p_4,\dots)$, která bude představovat počty k-úhelníků v grafu, definujme $\boldsymbol{P} = \lbrace k \mid p_k \neq 0 \rbrace$ jako její \textbf{množinu stěn}.

Klasifikujme posloupnosti $(p_k) = (p_3,p_4,\dots)$ podle jejich vlastností.

\begin{definice}[Neutrální posloupnost]\label{def01:neutralni}
Posloupnost $(p_k) = (p_3,p_4,\dots)$ nezáporných celých čísel je neutrální, pokud $\sum_{k \geq 3}{(6-k)p_k}=0$.
\end{definice}

\begin{definice}[Přípustná posloupnost]\label{def01:pripustna}
Posloupnost $(p_k) = (p_3,p_4,\dots)$ nezáporných celých čísel je přípustná, pokud $\sum_{k \geq 3}{(6-k)p_k}=6$.
\end{definice}

\begin{definice}[Realizovatelná posloupnost]\label{def01:realizovatelna}
Posloupnost $(p_k) = (p_3,p_4,\dots)$ nezáporných celých čísel je realizovatelná, pokud existuje konečný rovinný kubický graf, který má právě $p_k$ $k$-úhelníků.
\end{definice}

Všimněme si, že původní Eberhardova věta, a i mnoho jejích variant, byla formulována pro 3-souvislé grafy, tedy přesněji pro jednoduché konvexní 3-polytopy. Bijekci mezi těmito dvěma strukturami dokázal až o několik desítek let později Steinitz \cite{Steinitz}. 

Pro ujištění pojmů uveďme následující jednoduché vlastnosti neutrálních, přípustných a realizovatelných posloupností, které využijeme později.
\begin{tvrz}
Pro každou neutrální (či přípustnou nebo realizovatelnou) posloupnost $p=(p_3,p_4,\dots)$ existuje $h \in \mathbb{N}$, že $ \forall k \in \mathbb{N}, k>h $ platí $ p_k=0$.
\end{tvrz}

\begin{dukaz}
Neutrální posloupnost $p$ splňuje $\sum_{k \geq 3}{(6-k)p_k}=0$ neboli $\sum_{3 \leq k \leq 5}{(6-k)p_k}=\sum_{k \geq 5}{(k-6)p_k}$. Všechny sčítance na obou stranách výrazu jsou kladné. Hodnota levého součtu tří sčítanců je konečná. Pravý součet má tedy také konečnou hodnotu a navíc mají všechny jeho nenulové sčítance hodnotu alespoň 1. Nenulových hodnot v posloupnosti je tedy pouze konečně. 
\end{dukaz}

\begin{tvrz}\label{veta:posloupnosti}
\begin{description}
\item[] Pro každou $q$ neutrální posloupnost $\exists a,b \in Q : a <6 \wedge b>6$.
\item[] Součet dvou neutrálních posloupností je neutrální posloupnost. Součet neutrální a přípustné posloupnosti je přípustná posloupnost.
\item[] Pro každou přípustnou posloupnost $(p_k) = (p_3,p_4,\dots)$ existuje neutrální posloupnost $q$, že $p+qn$ je realizovatelná pro nějaké $n \in \mathbb{N}$.
\end{description}
\end{tvrz}

\begin{dukaz}
Pro ukázání první vlastnosti si stačí uvědomit, že chceme $\sum_{k \geq 3}{(6-k)p_k} = 0$, kde pro $k < 6$ je sčítaný člen kladný, zato pro $k>6$ je sčítaný člen záporný. Druhá vlastnost plyne z definice a distributivity násobení. Třetí tvrzení pro je slabší verze Věty \ref{veta:Eberhard}.
\end{dukaz}
K poslední vlastnosti se nabízí otázka, jak lze $q$ omezit, aby pořád $p+qn$ byla realizovatelná. Eberhard ukázal, že stačí  $q = (0,0,0,q_6=1,0,\dots)$, tedy přidat šestiúhelníky. DeVos a kol. ukázal že i omezení na  $q = (0,0,1,0,1,0,\dots)$ funguje. V této práci ukážeme, že posloupností s pouze dvěma nenulovými hodnotami, které v tomto směru vyhovují, existuje výrazně víc.



