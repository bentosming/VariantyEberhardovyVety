\chapter*{Závěr}
\addcontentsline{toc}{chapter}{Závěr}
V práci se podařilo dokončit důkaz několika instancí Hypotézy \ref{veta02:hypoteza}, konkrétně pro neutrální posloupnosti, které mají jen dvě nenulové hodnoty, velikosti jejích stěn jsou nesoudělné a dostatečně malé. Výčet všech dvojic těchto hodnot je v~Tabulce~\ref{obr03:tabvysledky}.

Kromě tohoto teoretického výsledku poskytujeme implementaci algoritmů z Kapitoly \ref{resitko}, který může sloužit při hledání 3-regulárních grafů, speciálně i s konkrétně zadanými velikostmi stěn.

Na základě získaných výsledků bychom v budoucnu rádi zjistili, jestli Hypotéza \ref{veta02:hypoteza} platí i pro nějaké posloupnosti, jejichž množina stěn má soudělné prvky. Také bychom rádi ověřili, že pokud jsou nesoudělné, platí hypotéza vždy.

Vedle hlavního tématu práce bychom rádi získali vhodnější algoritmy pro kreslení rovinných grafů do lidsky dobře čitelné podoby.