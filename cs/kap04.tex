%%% Fiktivní kapitola s instrukcemi k PDF/A

\chapter{Výsledky}

Díky programu popsanému v předchozí kapitole bylo možné zkusit dokončit důkaz věty TODO odkaz pro některé neutrální posloupnosti $q$ (na volbě posloupnosti $p$ nezáleží, protože konstrukce pro jádrový triark nezávisí na velikosti jádra, více viz TODO odkaz). 

Vzhledem k výpočetním omezením programu jsme zkusili dokončit důkaz věty pouze pro takové neutrální posloupnosti $q$, že $Q = {r, s}$ a navíc $r<s<18$. Seznam posloupností, pro které program nalezl potřebné grafy je v tabulce níže TODO odkaz?. Pokud bychom se omezili na rozsah $r<s<14$, pak program grafy nalezl právě tehdy, když $r$ a $s$ jsou nesoudělná čísla.

\begin{figure}[h]\centering
\begin{tabular}{ c c c c c c c c c c c c }
  - & 7 & 8 & 9 & 10 & 11 & 12 & 13 & 14 & 15 & 16 & 17 \\
  3 & $\bullet$ & $\bullet$ &  & $\bullet$ & $\bullet$ &  & $\bullet$ & $\bullet$ &  & $\bullet$ & $\bullet$ \\
  4 & $\bullet$ &  & $\bullet$ &  & $\bullet$ &  & $\bullet$ &  & $\bullet$ \\
  5 & $\bullet$ & $\bullet$ & $\bullet$ &  & $\bullet$ & $\bullet$ & $\bullet$  
\end{tabular}
\caption{Úplný výčet dvojic stran, pro které se podařilo dokončit důkaz věty TODO odkaz.}
\label{obr03:Nvyber}
\end{figure}
TODO označneí

Všechny potřebné grafy pro doložení důkazu jsou k práci přiloženy. TODO přiložit

Přirozenou snahou při zkoumání nalezených grafů je grafy zobrazit, aby byly pro člověka dobře čitelné. V další sekci se tomuto tématu krátce věnujeme. 

\section{Kreslení}

TODO přesunout
Při hledání triarků byla vynaložena netriviální snaha pro nalezení způsobu, jak výsledný graf dobře zobrazit - od rovinného grafu by se dalo čekat, že nepůjde o příliš složitý úkol. Po špatné zkušenosti s dostupnými možnostmi jako GraphViz nebo funkcemi v SageMath, došlo na implementaci kreslícího algoritmu založeného na Tuttově kreslení TODO zmínit odkaz a dál o kreslení nemluvit? Nebo popsat? 
